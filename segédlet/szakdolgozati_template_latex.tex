\documentclass[a4paper,12pt]{report}
\usepackage[T1]{fontenc}
\usepackage[utf8]{inputenc}
\def\magyarOptions{defaults=hu-min}
\usepackage[magyar]{babel}
\usepackage{amsthm, amssymb,amsmath,hyperref}
\usepackage{enumerate, graphicx, xcolor}

\usepackage{chngcntr}
\counterwithout{figure}{chapter}
\counterwithout{figure}{section}
\counterwithout{figure}{subsection}
%\usepackage{pgf,tikz,float}
%\usepackage{tikzlings}
%\usepackage{tikzducks}
%\usetikzlibrary{arrows}
\usepackage[nobysame]{amsrefs}
%\usepackage{amsmath}

\usepackage{geometry}
 \geometry{
 a4paper,
 total={160mm,247mm},
 left=25mm,
 top=25mm,
 }


\newtheorem{theo}{tétel}[section]
\newtheorem{defin}[theo]{definíció}
\newtheorem{lemma}[theo]{lemma}
\newtheorem{all}[theo]{állítás}
\newtheorem{kov}[theo]{következmény}

\theoremstyle{definition}
\newtheorem{definition}[theo]{definíció}

\theoremstyle{remark}
\newtheorem{megj}[theo]{megjegyzés}




\date{today}

\linespread{1.3}



\begin{document}
\thispagestyle{empty}

\begin{center}
 {\Large A dolgozat formai követelményei}   
\end{center}
 
\vspace{1 cm}

{\bf Ajánlott oldalszám:}
\begin{itemize}
\item BSc szakdolgozat: 20-25 oldal, 
\item MSc diplomamunka: 35-50 oldal
\item Tanárszakos szakdolgozat: 50000-80000 karakter szóközökkel (lásd. TKK szabályzat)
\end{itemize}

\vspace{1 cm}

{\bf Másfeles sorköz, sorkizárt, 12-es betű méret (általában Computer Modern), margók: jobb, bal, lent és fent 2,5 cm} (ahogy ez a template van beállítva, az jó).

\vspace{1 cm}

{\bf Kötelező elemei a dolgozatnak}
\begin{itemize}
    \item Címlap
    \item Tartalmi összefoglaló (kivonat)
    \item Tartalomjegyzék
    \item \textcolor{red}{Érdemi rész}
(Szakterület specifikus fejezeteket tartalmaz, kérjük, konzultáljon a témavezetővel, hogy a példában közzétett kísérletes tudományterületeken használt fejezetekből az Ön dolgozatában melyikre van szükség!)
\item Irodalomjegyzék
\item Nyilatkozat
\end{itemize}
\newpage



\pagenumbering{roman}

%Elso  oldal 
\thispagestyle{empty}

\begin{center}
\vspace*{0.2cm} {\Large\bf Szegedi Tudományegyetem}
\vspace{0.3cm}

{\Large\bf Természettudományi és Informatikai Kar}
\vspace{0.3cm}

{\Large\bf XXXXXXX Intézet, XXXXXXXXX Tanszék}
\vspace{3cm}



{\Large SZAKDOLGOZAT/DIPLOMAMUNKA}
% BSc és tanárszak esetén, szakdolg., MSc esetén diplomamunka

\vspace*{1.5cm}

{\LARGE\bf A szakdolgozat címe}

opcionálisan, formázatlanul
A szakdolgozat angol címe 



\vspace*{4cm}

{\large
\begin{tabular}{c@{\hspace{2cm}}c}
\emph{Készítette:}     &\emph{Témavezető:}\\
\bf{Végzős Edömér}  &\bf{Dr. Kiváló Kitti}\\
XXXXXX BSc hallgató    & egyetemi docens\\
&
\end{tabular}
}

\vspace*{1,5cm}

{\Large Szeged\\ \vspace{2mm} 20XX}
\end{center}

%masodik oldal osszefogalalo
\begin{abstract}
A dolgozat tartalmának rövid (max. 1 oldal) összefoglalása. A következő részekből áll: rövid irodalmi összefoglaló, a dolgozat elkészítéséhez használt módszerek, eredmények, konklúzió

{\bf Kulcsszavak:} a dolgozat tartalmára specifikusan jellemző 4-6 szó, egymástól vesszővel elválasztva
\end{abstract}



\newpage


\pagebreak

\tableofcontents
\pagebreak
%\listoffigures
%\pagebreak








\chapter{Bevezetés}
\pagenumbering{arabic}

(a fejezet új oldalon kezdve)

Tartalmazza a problémafelvetést, a témaválasztás indoklását és a munka célját.

A célkitűzések részletes megfogalmazása külön fejezetben az irodalmi áttekintés után legyen.




\chapter{Irodalmi áttekintés}

(a fejezet új oldalon kezdve) 


Alfejezetekre osztható.

\section{Alfejezet}
\subsection{Al-alfejezet}
\subsection{Al-alfejezet}
\section{Alfejezet}



A dolgozat témájához kapcsolódó korábbi eredmények ismertetése, a rendelkezésre álló szakirodalmi adatok összefoglalása, elemzése. 
A felhasznált irodalmat a szövegben hivatkozni kell (zárójelben) vagy [kapcsos zárójelben]. A hivatkozás történhet csak számozással \cite{ALE3} vagy az első szerző nevével és a megjelenés évszámával pl. (több szerző esetén:  Gipsz és mtsi 1963 \cite{WSB}; két szerző esetén: Monroe és O’Donell, 2001 \cite{MD}). A hivatkozott irodalmat a dolgozat végén az irodalomjegyzékben össze kell gyűjteni olyan módon, hogy mások számára is fellelhető legyen!

Tartalmazhat szakirodalmi ábrákat. Ezeket középre kell rendezni és hivatkozni kell rájuk a szövegben (lásd \ref{fig:korok}.). A hivatkozások, akár irodalmi, akár ábrahivatkozás, részei a mondatnak!

\begin{figure}
\centering
\includegraphics[height=7 cm]{b4918.png}
\caption{ \label{fig:korok} Az ábra címe (az ábraaláírást mindig az ábra alá kell elhelyezni!)
Szükség esetén az ábra eredetére vonatkozó hivatkozás.
}
\end{figure}

\chapter{Célkitűzések}

Ebből látszik, hogy a korábbi eredményekhez képest mit szeretnének csinálni.

\chapter{Felhasznált anyagok és eszközök}

(a fejezet új oldalon kezdve) 

Szakirodalmi feldolgozás esetén nem releváns!

A következő fejezettel (Alkalmazott módszerek) összevonható.

\chapter{Alkalmazott módszerek}

(a fejezet új oldalon kezdve) 

Szakirodalmi feldolgozás esetén nem releváns!


Alfejezetekre osztható. 

Olyan részleteséggel kell megírni, hogy mások számára megismételhetőek legyenek.

\chapter{Eredmények}

(a fejezet új oldalon kezdve) 

Az elért eredmények világos és részletes leírását tartalmazza. 

Egyértelműen különüljön el a hallgató saját munkájának eredménye az irodalmi áttekintéstől, elméleti alapokat tartalmazó részektől.

Alfejezetekre osztható.

\section{Táblázatok, képletek, egyéb értéssegítők}

Tartalmazhat táblázatokat, melyekre hivatkozni kell a szövegben, például \ref{table:2}. táblázat. A táblázatot középre kell rendezni és a szövegben lévő hivatkozás közelében kell elhelyezni.

\begin{table}[h!]
\centering
\caption{A táblázat címe a táblázat fölé kerül.}
\label{table:2}
\vspace{.2 cm}
\begin{tabular}{||c c c c||} 
 \hline
 Col1 & Col2 & Col2 & Col3 \\ [0.5ex] 
 \hline\hline
  4 & 545 & 18744 & 7560 \\
 5 & 88 & 788 & 6344 \\ [1ex] 
 \hline
\end{tabular}
\end{table}




Tartalmazhat ábrákat. Ezeket középre kell rendezni és hivatkozni kell rájuk a szövegben (lásd \ref{fig:korokuj}). Ha az Irodalmi áttekintés fejezet tartalmazott ábrá(ka)t, akkor a számozás ebben a fejezetben nem újra kezdődik, hanem folytatódik.

\begin{figure}
\centering
\includegraphics[height=7 cm]{b4918.png}
\caption{ \label{fig:korokuj} Az ábra címe mindig a kép alatt legyen.
}
\end{figure}

Egész oldalas ábrák mellékletben legyenek elhelyezve!

Használjuk bátran a \LaTeX  brilliáns egyenletszerkesztő rendszerét!

A szövegközi $x^2$ képletetket nem, de a kiemelt formulákat általában számozzuk:

 \begin{equation}\label{circle-area}
\mathrm{Var} (A(K_n^r)))\ll n^{-2},
\end{equation}

kivéve a hosszabb számolásokat

\begin{align}
E (& F_n(x_{n+1})^2) = \frac 1{A(K)^{n+1}}\int_K\int_{K^n} \left (\sum_I \mathbf I (F_I\in\mathcal F_{n}(x_{n+1}))\right )^2 d X_n d x_{n+1}\nonumber\\
& = \frac 1{A(K)^{n+1}} \int_K\int_{K^n} \left (\sum_I \mathbb I (F_I\in\mathcal F_{n}(x_{n+1}))\right ) \nonumber \\
& \quad \quad \times \left (\sum_J \mathbf I (F_J\in\mathcal F_{n}(x_{n+1}))\right ) d X_n d x_{n+1}\nonumber\\
&\leq \frac 1{A(K)^{n+1}}\sum_I\sum_J \int_K\int_{K^n} \mathbf I (F_I\in\mathcal F_{n}(x_{n+1}))
\mathbf I (F_J\in\mathcal F_{n}(x_{n+1}))\nonumber\\
&\quad\quad\times\mathbf I(d_H(K,K_n)\leq\varepsilon_K) d X_n d x_{n+1}
+O((1-c_K)^n)\label{ineq1}
%\label{Fn-first} & \ll \frac 1{A(K)^{n+1}} \sum_I\sum_J \int_K\cdots \int_K \indi (F_I\in\F_{n}(x_{n+1}))
%\indi (F_J\in\F_{n}(x_{n+1}))\\ %\notag\\
%&\quad\quad\times\indi(d_H(K,K_n)\leq\varepsilon_K) \d X_n \d x_{n+1} %\textcolor{red}{\sout{+O((1-c_K)^n)}}
%\notag
\end{align}
A képletben használt mennyiségek legyenek megadva az első használatuknál.
\chapter{Összefoglalás}

(új oldalon kezdve)

A dolgozat eredményeinek összefoglalása, következtetések levonása.

Az összefoglalásban egyértelműen jelezve legyen a hallgató saját szerepe/eredményei.





\newpage
\begin{bibdiv}
\begin{biblist}
 \addcontentsline{toc}{chapter}{Irodalomjegyzék}


\bib{ALE3}{article}{
author={A. L. Edmonds},
title={The geometry of an equifacetal simplex},
journal={Indiana University Reprint},
date={2003},
}


\bib{AC64}{book}{
author={N. Altshiller-Court},
title={Modern Pure Solid Geometry},
publisher={Chelsea Publishing Co., N. Y.},
date={1964},
}


\bib{WSB}{article}{
author={J. Gipsz},
author={M. Kalányos},
author={Z. Kovács},
title={The Chasm},
journal={Doom II.},
volume={76},
date={1963},
pages={661–669}
}

\bib{MD}{article}{
author={J. Monroe},
author={M. O'Donnel},
title={The kiss},
journal={Amer. Math. Monthly},
volume={76},
date={1969},
pages={661–663}
}
\end{biblist}
\end{bibdiv}


\newpage
{\Huge \bf Köszönetnyilvánítás}

 \addcontentsline{toc}{chapter}{Köszönetnyilvánítás}

\vspace{2 cm}

(nem kötelező elem), (új oldalon kezdve) 

Ebben a fejezetben lehet köszönetet mondani mindazoknak, akik segítették a dolgozat elkészülését. Itt lehet megemlíteni továbbá a munkát támogató pályázatokat, ösztöndíjakat, stb.

\newpage
{\Huge \bf Nyilatkozat}

 \addcontentsline{toc}{chapter}{Nyilatkozat}

\vspace{2 cm}

{\it A szöveg kötött, kérjük ezt használni!}

Alulírott, Végzős Edömér, xxxx szakos hallgató, kijelentem, hogy a szakdolgozatban ismertetettek saját munkám eredményei, és minden felhasznált, nem saját munkából származó eredmény esetén hivatkozással jelöltem annak forrását. 


\begin{flushleft}
\vspace*{1cm}
Szeged, \today
\end{flushleft}

\begin{flushright}
 \vspace*{1cm}
 \makebox[7cm]{\rule{6cm}{.4pt}}\\
 \makebox[7cm]{\emph{Végzős Edömér}}
\end{flushright}

\pagebreak

\newpage
{\Huge \bf Mellékletek}

 \addcontentsline{toc}{chapter}{Mellékletek}

\vspace{2 cm}

(nem kötelező elem, a dolgozat oldalszámaiba nem tartozik bele.), (új oldalon kezdve)

Ebben a fejezetben lehet elhelyezni a nagyobb táblázatokat, ábrákat, adathalmazokat.


\end{document}



